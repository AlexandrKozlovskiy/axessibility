\documentclass[11pt,a4paper]{article}
%\usepackage{geometry}
%
%\geometry{left=2.15in, right=2.15in, top=1.72in, bottom=1.72in}

\usepackage{amsmath,amsthm}
\usepackage{amssymb}
\usepackage{hyperref}
\usepackage[english]{babel}

%\usepackage{documentation}

\title{The \textbf{axessibility} package}
\author{Dragan Ahmetovic, Tiziana Armano, Cristian Bernareggi, Michele Berra, \\
 Anna Capietto, Sandro Coriasco, Nadir Murru, Alice Ruighi\\
(Department of Mathematics G. Peano, University of Turin)}

\begin{document}

\maketitle

\begin{abstract}
PDF documents containing formulae generated by \LaTeX\ are usually not accessible by assistive technologies for visually impaired people (i.e., by screen readers and braille displays). The package manages this issue allowing to create a PDF document where the formulae are read by these assistive technologies, since it automatically generates hidden comments in the PDF document (by means of the /ActualText attribute) in correspondence to each formula. The package does not generate PDF/UA. Status is in progress.
\end{abstract}

\tableofcontents

%QUESTIONE LICENZA, ACKNOWLEDGMENT E CITAZIONE

\section{Introduction}
This package focuses on the specific problem of the accessibility of PDF documents generated by \LaTeX\ for visually impaired people.
When a PDF document is generated starting from \LaTeX\, formulae are not accessible by screen readers and braille displays. They can be made accessible by inserting a hidden comment, i.e., an actual text, similarly to the case of web pages. This can be made, e.g., by using the \LaTeX\ package pdfcomment.sty. In any case, this task must be manually performed by the author and it is surely inefficient, since the author should write the formulae and, in addition, insert a description for each formula. Note also that the package pdfcomment.sty does not allow to insert special characters like `backslash', `brace', etc, in the comment. Moreover, with these solutions, the reading is bothered since the screen reader reads incorrectly the formula and then the correct comment of the formula. There are also some \LaTeX\ packages that try to improve the accessibility of PDF documents produced by \LaTeX\. In particular the packages accsupp.sty and accessibility$\_$meta.sty has been developed in order to obtain tagged PDF documents. However, both packages do not solve the problem of the accessibility of formulae. The package accsupp.sty develops some interesting tools for commenting formulae using also special characters (possibility that is not available in the pdfcomment.sty package). Moreover, this is not an automatized method, since the comment must be manually inserted by the author. The package accessibility$\_$meta.sty is an improved version of the package accessibility.sty. This package allows the possibility of inserting several tags for sections, links, figures and tables. However, even if these tags are recognized by the tool for chekcing tags of Acrobat Reader Pro, they are not always recognized by the screen readers. Moreover, this package does not manage formulae. Our package automatically produces an actual text corresponding to the \LaTeX\ commands that generate the formulae. This actual text is hidden in the PDF document but the screen reader reads it without reading  any incorrect sequence before. 


\section{Prerequisites}

The package \textbf{axessiblity} requires the following packages: \textbf{accsupp, amsmath, amssymb}.

\section{Package specification}

If you use \LaTeX\ $2_\epsilon$ simply do the following \\ \\
\indent $\backslash$\textbf{usepackage}$\{$\textbf{axessibility}$\}$
\\ \\
The package includes the following features:
\begin{itemize}
\item The commands \\ \\
\indent $\backslash$\textbf{pdfcompresslevel=0} \\
\indent $\backslash$\textbf{pdfoptionpdfminorversion=6} \\ \\
that produce an uncompressed PDF document. 
\item The command \\ \\
\indent $\backslash$\textbf{BeginAccSupp} 
\\ \\
contained in the package \textbf{accsupp} has been redefined so that the screen readers access the actual text created by this command. 
\item The new command \\ \\
\indent $\backslash$\textbf{wrap$\#$1}
\\ \\
allows to store its input into an actual text in the PDF document (e.g., the \LaTeX\ commands for generating a formula).

\item The environments \\ \\
\indent $\backslash$\textbf{begin}$\{$\textbf{equation}$\}$, $\backslash$\textbf{end}$\{$\textbf{equation}$\}$ \\
\indent $\backslash$\textbf{begin}$\{$\textbf{equation*}$\}$, $\backslash$\textbf{end}$\{$\textbf{equation*}$\}$ \\
\indent $\backslash$[, $\backslash$]\\
\indent $\backslash$(, $\backslash$)\\ %DA SISTEMARE GRAFICAMENTE

has been redefined. In each environment above listed, the command $\backslash$\textbf{wrap} is inserted together to the command $\backslash$\textbf{collect@body} so that all the content of the environment is automatically stored into an ActualText in the PDF document.
\end{itemize}

\section{Usage}

\noindent An author that would like to create an accessible PDF document for visually impaired people can add this package using the above environments for inserting the formulae. The \LaTeX\ code of the inserted formulae will be added as hidden comments in correspondence to the location of the formulae in the text. This will allow the user to access the formula code with the screen reader and with the braille refreshable display. Additionally, the package enables to copy the formula \LaTeX\ code from the PDF reader and paste it elsewhere.

\noindent Note that, to preserve the compatibility with Acrobat Reader, our package discourages the use of the underscore character ($\_$) which is not correctly read using screen readers in combination with this PDF reader. Alternatively, we suggest to use the equivalent command $\backslash$\textbf{sb}.

\noindent Inlined and display mathematical modes ($\$$, $\$\$$) are not supported in this version of the package. However external scripts provided as companion software described in the following section can also address these use cases. 

\noindent While multiline environments \\ \\
\indent $\backslash$\textbf{begin}$\{$\textbf{align}$\}$, $\backslash$\textbf{begin}$\{$\textbf{eqnarray}$\}$, $\backslash$\textbf{begin}$\{$\textbf{multline}$\}$
\\ \\
are not directly supported, it is possible to use \\ \\
\indent $\backslash$\textbf{begin}$\{$\textbf{equation}$\}$$\backslash$\textbf{begin}$\{$\textbf{aligned}$\}$ 
\\ \\
for typesetting multiline formulae.

\section{External scripts and screen reader integration}

In addition to the package, we also provide scripts that complement package functionalities. 

\subsection{Preprocessing scripts}
While we warmly suggest to follow the indications provided in the usage guide (suggested commands and environments), it is also possible to apply our package to an already existing \LaTeX\ document. In this case, it is necessary to preprocess the document in order to replace the unsupported commands and environments with the suggested ones. We provide a preprocessing script for this use case at our Github repository.
% TODO: COMPLETARE CON LINK ECC...

\subsection{Expansion of user macros}
Note that, custom macros used by the author within the formulae are copied as-is into the actual text in the hidden comment. This macros may bear no meaning for other readers, so it may be more meaningful to expand those macros into the original \LaTeX\ commands. We provide a script that can parse \LaTeX\ document and replace all the user macros within the formulae with their expanded definitions. You can download this script at our Github repository.
% TODO: COMPLETARE CON LINK ECC...

\subsection{Screen reader dictionaries}
\LaTeX\ commands that are included as actual text in the hidden comments corresponding to formulae may appear awkward when read by the screen reader. We provide dictionaries for JAWS and ?NVDA? screen readers that convert \LaTeX\ commands into natural language. Please note that the braille refreshable display will still show the formulae in their original \LaTeX\ representations. The dictionaries can be downloaded at our Github repository.
% TODO: COMPLETARE CON LINK ECC...



\end{document}